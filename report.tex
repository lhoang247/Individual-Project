\documentclass[12pt]{report}
\usepackage{lipsum}
\usepackage{mathtools}
\usepackage{array}
\usepackage{colortbl}
\usepackage{chemfig}
\RequirePackage[left=3cm,top=1.5cm,right=3cm,bottom=2.5cm,nohead,nofoot]{geometry}

\begin{document}

\thispagestyle{empty}   % this page does not have a header

\tableofcontents

\chapter{Introduction}

\chapter{Output Summary}

\chapter{Background Research}


\section{What is computer vision?}


Computer vision (CV) is a scientific discipline that studies how computer scan efficiently perceive, process, and understand information from visual data such as images and videos.

As humans, we can classify three-dimensional objects with ease, whether the pictures are the same object with different colours or angles, we are good at determining the object we are classifying. 
Computer vision has been developed to detect edges from a pixelated image, face detection, and has been used to develop 3D models from a snapshot yet the technology we have today could be compared to a young child's biological vision. 

Computer vision is used in various real world application such as traffic surveillance or medical imaging (SZELISKI, 2020), where people are now able to utilize magnetic resonance imaging (MRI) to safely analysis the heart wall motion where the end result is a 3d model of the heart pumping (Metaxas, 1997). 

In recent years, computer vision has been adapting deep learning algorithms to efficiently classify unseen objects within pixel images and videos. 


\section{Deep learning}

Deep learning uses artificial intelligence (AI) to try and simulate the choices that a human brain will make. Problems that have regression or classification outputs can be solved by passing data/inputs through artificial neurons which were previously tweaked for the specific problem by training data. There are many different variants of deep learning algorithms such as Artificial Neural Networks (ANN) or Long Short-Term Memory (LSTM) Networks (Hochreiter, 1997) which build onto each other.

\subsection{Neural networks}

Neural networks is a concept more than a algorithm.

\subsection{Convolutional neural networks}

\section{Existing systems}



\end{document}